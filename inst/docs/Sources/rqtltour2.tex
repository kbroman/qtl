\documentclass[10pt,letterpaper]{article}

\usepackage{times}     % times font
\usepackage{color}     % getting colored text
\usepackage{amsmath}
\usepackage{hyperref}

% revise margins
\setlength{\headheight}{0.0in}
\setlength{\topmargin}{-0.25in}
\setlength{\headsep}{0.0in}
\setlength{\textheight}{9.5in}
\setlength{\footskip}{0.35in}
\setlength{\oddsidemargin}{-0.25in}
\setlength{\evensidemargin}{-0.25in}
\setlength{\textwidth}{7.0in}
\setlength{\parindent}{0pt}
\setlength{\parsep}{12pt}

% font colors
\newcommand{\usercolor}{\color [named]{BlueViolet}}
\newcommand{\othercolor}{\color [named]{Mahogany}}
\newcommand{\lod}{\text{LOD}}

% environment for references
\newenvironment{hanging}
{\begin{list}{}
        {\setlength{\labelwidth}{0in}
         \setlength{\leftmargin}{1em}
         \setlength{\itemindent}{-1em}
        }
}
{\end{list}}


\begin{document}

\begin{center}
\rule{7.0in}{1mm} \vspace{0mm}

{\Large \textbf{A shorter tour of R/qtl}} \vspace{4mm}

{\large Karl W Broman} \vspace{2mm}

Department of Biostatistics and Medical Informatics\\
University of Wisconsin -- Madison

\vspace{2mm}
\href{http://www.rqtl.org}{http://www.rqtl.org}
\vspace{2mm}

21 March 2012 % the date

\rule{7.0in}{1mm} 
\end{center}



\textbf{Preliminaries} \vspace{6pt}

\begin{enumerate}

\item To install R/qtl, type (within R) {\usercolor \verb-install.packages("qtl")-}
(This needs to be done just once.)

\item To load the R/qtl package, type

\usercolor \verb|library(qtl)| \normalcolor

This needs to be done every time you start R. (There is a way to
have the package loaded automatically every time, but we won't discuss
that here.)

\item To view the objects in your workspace:

\usercolor \verb|ls()| \normalcolor

\item The best way to get help on the functions and data sets in R
(and in R/qtl) is via the html version of the help files. One way to
get access to this is to type 

\usercolor \verb-help.start()- \normalcolor

This should open a browser with the main help menu.  
If you then click on \othercolor Packages \normalcolor $\rightarrow$
\othercolor qtl\normalcolor , you can see all of the available
functions and datasets in R/qtl.  For example, look at the help file
for the function \verb-read.cross-.

An alternative method to view this help file is to type one of the
following: 

\usercolor \verb|help(read.cross)| \\
\verb|?read.cross| \normalcolor

The html version of the help files are somewhat easier to read, and
allow use of hotlinks between different functions.  

\item All of the code in this tutorial is available as a file from
  which you may copy and paste into R, if you prefer that to typing.
  Type the following within R to get access to the file:

\usercolor \verb-url.show("http://www.rqtl.org/rqtltour2.R")-
\normalcolor

\end{enumerate}


\vspace{12pt}
\textbf{Data import} \vspace{6pt}

We will consider data from Sugiyama et al., Physiol Genomics 10:5--12,
2002. 
The
data are from an intercross between BALB/cJ and CBA/CaJ; only male
offspring were considered.  There are four phenotypes: blood pressure,
heart rate, body weight, and heart weight.  We will focus on the blood
pressure phenotype, will consider just the 163 individuals with
genotype data and, for simplicity, will focus on the autosomes.
The data are contained in the comma-delimited file
\href{http://www.rqtl.org/sug.csv}{http://www.rqtl.org/sug.csv}.  

\begin{enumerate}
\addtocounter{enumi}{5}

\item Load the data into R/qtl as follows.

\usercolor 
\verb|sug <- read.cross("csv", "http://www.rqtl.org", "sug.csv",| \\
\verb|                  genotypes=c("CC", "CB", "BB"), alleles=c("C", "B"))|
\normalcolor
\end{enumerate}

The function \verb-read.cross- is for importing data into R/qtl.
\verb-"sug.csv"- is the name of the file, which we import directly
from the R/qtl website.  \verb-genotypes- indicates the codes used for
the genotypes; \verb-alleles- indicates single-character codes to be
used in plots and such.
\vspace{12pt}

\verb-read.cross- loads the data from the file and formats it into
a special cross object, which is then assigned to \verb-sug- via the
assignment operator (\verb:<-:).  


\clearpage
\textbf{Diagnostics} \vspace{6pt}

Generally, at this point, one would spend considerable time studying
the genotype and phenotype data, looking for potential errors.  In
many cases, about half of the analysis time is devoted to such
diagnostics.  

\vspace{12pt}

In previous tutorials, we've often gotten bogged down in this part,
and so we'll skip it here, assume that the data are okay, and
jump right into QTL mapping.  See the longer (``brief'') tour of R/qtl
at
\href{http://www.rqtl.org/tutorials}{http://www.rqtl.org/tutorials},
or Chapter 3 of Broman and Sen (2009).


\vspace{12pt}
\textbf{Summaries} \vspace{6pt}
\nopagebreak

The data object \verb-sug- is complex; it contains the genotype
data, phenotype data and genetic map.  R has a certain amount of
``object oriented'' facilities, so that calls to functions like
\verb-summary- and \verb-plot- are interpreted appropriately for the object
considered.  

The object \verb-sug- has ``class'' \verb-"cross"-, and so calls to
\verb-summary- and \verb-plot- are actually sent to the functions
\verb-summary.cross- and \verb-plot.cross-.

\begin{enumerate}
\addtocounter{enumi}{6}
\item Get a quick summary of the data.  (This also performs a variety
  of checks of the integrity of the data.)

\usercolor
\verb|summary(sug)|
\normalcolor

We see that this is an intercross with 163 individuals.  There are 6
phenotypes, and genotype data at 93 markers across the 19 autosomes.
The genotype data is quite complete.

\item There are a number of simple functions for pulling out pieces of summary
  information.  Hopefully these are self-explanatory.

\usercolor
\verb|nind(sug)| \\
\verb|nchr(sug)| \\
\verb|totmar(sug)| \\
\verb|nmar(sug)| \\
\verb|nphe(sug)|
\normalcolor



\item Get a summary plot of the data.

\usercolor
\verb|plot(sug)|
\normalcolor

The plot in the upper-left shows the pattern of missing genotype data, with
black pixels corresponding to missing genotypes.  The next plot shows
the genetic map of the typed markers.  The following plots are
histograms or bar plots for the six phenotypes.  The last two
``phenotypes'' are sex (with 1 corresponding to males) and mouse ID.


\item Individual parts of the above plot may be obtained as follows.

\usercolor
\verb|plotMissing(sug)| \\
\verb|plotMap(sug)| \\
\verb|plotPheno(sug, pheno.col=1)| \\
\verb|plotPheno(sug, pheno.col=2)| \\
\verb|plotPheno(sug, pheno.col=3)| \\
\verb|plotPheno(sug, pheno.col=4)| \\
\verb|plotPheno(sug, pheno.col=5)| \\
\verb|plotPheno(sug, pheno.col=6)|
\normalcolor


\end{enumerate}

\vspace{12pt}
\textbf{Single-QTL analysis} \vspace{6pt}

Let's now proceed to QTL mapping via a single-QTL model.  

\begin{enumerate}
\addtocounter{enumi}{10}
\item We first calculate the QTL genotype probabilities, given the
observed marker data, via the function \verb-calc.genoprob-.  This is
done at the markers and at a grid along the chromosomes.  The argument
\verb-step- is the density of the grid (in cM), and defines the
density of later QTL analyses.

\usercolor
\verb|sug <- calc.genoprob(sug, step=1)|
\normalcolor

The output of \verb-calc.genoprob- is the same cross object as input,
with additional information (the QTL genotype probabilities) inserted.  We
assign this back to the original object (writing over the previous
data), though it could have also been assigned to a new object.

\item To perform a single-QTL genome scan, we use the function
  \verb-scanone-.  By default, it performs standard interval mapping
  (that is, maximum likelihood via the EM algorithm).  Also, by
  default, it considers the first phenotype in the input cross object
  (in this case, blood pressure).

\usercolor
\verb|out.em <- scanone(sug)|
\normalcolor

\item The output has ``class'' \verb-"scanone"-.  The \verb-summary-
  function is passed to the function \verb-summary.scanone-, and gives
  the maximum LOD score on each chromosome.

\usercolor
\verb|summary(out.em)|
\normalcolor

\item Alternatively, we can give a threshold, e.g., to only see those
  chromosomes with LOD $>$ 3.

\usercolor
\verb|summary(out.em, threshold=3)|
\normalcolor

\item We can plot the results as follows.

\usercolor
\verb|plot(out.em)|
\normalcolor

\item We can do the genome scan via Haley-Knott regression by calling
  \verb-scanone- with the argument \verb-method="hk"-.  

\usercolor
\verb|out.hk <- scanone(sug, method="hk")|
\normalcolor

\item We may plot the two sets of LOD curves together in a single call
  to \verb-plot-.

\usercolor
\verb|plot(out.em, out.hk, col=c("blue", "red"))|
\normalcolor

\item Alternatively, we could do the following:

\usercolor
\verb|plot(out.em, col="blue")| \\
\verb|plot(out.hk, col="red", add=TRUE)| 
\normalcolor

\item It's perhaps more informative to plot the differences:

\usercolor
\verb|plot(out.hk - out.em, ylim=c(-0.3, 0.3), ylab="LOD(HK)-LOD(EM)")|
\normalcolor
 
\item To perform a genome scan by the multiple imputation method, one
  must first call \verb-sim.geno- to perform the multiple
  imputations.  This is similar to \verb-calc.genoprob-, but with an
  additional argument, \verb-n.draws-, indicating the number of
  imputations.  We then call \verb-scanone- with \verb-method="imp"-.

\usercolor
\verb|sug <- sim.geno(sug, step=1, n.draws=64)| \\
\verb|out.imp <- scanone(sug, method="imp")|
\normalcolor

\item We may plot all three curves together as follows.

\usercolor
\verb|plot(out.em, out.hk, out.imp, col=c("blue", "red", "green"))|
\normalcolor

\item We can plot the LOD curves for just chromosomes 7 and 15 as follows.

\usercolor
\verb|plot(out.em, out.hk, out.imp, col=c("blue", "red", "green"), chr=c(7,15))|
\normalcolor

\item We can also look at differences.

\usercolor
\verb|plot(out.imp - out.em, out.hk - out.em, col=c("green", "red"), ylim=c(-1,1))|
\normalcolor

\end{enumerate}


\vspace{12pt}
\textbf{Permutation tests} \vspace{6pt}
\nopagebreak

To perform a permutation test, to get a genome-wide significance
  threshold or genome-scan-adjusted p-values, we use \verb-scanone-
  just as before, but with an additional argument, \verb-n.perm-,
  indicating the number of permutation replicates.  It's quickest to
  use Haley-Knott regression.

\begin{enumerate}
\addtocounter{enumi}{23}

\item In case the time to perform the permutation test is too long, you can
skip it (here) and load the results (that I calculated previously) for
this plus other time-consuming stuff we'll see shortly as follows.\label{various}

\usercolor
\verb|load(url("http://www.rqtl.org/various.RData"))|
\normalcolor


\item The code to do the actual permutation test is the following:

\usercolor
\verb|operm <- scanone(sug, method="hk", n.perm=1000)|
\normalcolor

\item A histogram of the results (the 1000 genome-wide maximum LOD
  scores) is obtained as follows:

\usercolor
\verb|plot(operm)|
\normalcolor

\item Significance thresholds may be obtained via the \verb-summary-
  function:

\usercolor
\verb|summary(operm)| \\
\verb|summary(operm, alpha=c(0.05, 0.2))|
\normalcolor

\item Most importantly, the permutation results may be used along with
  the \verb-scanone- results to have significance thresholds and
  p-values calculated automatically:

\usercolor
\verb|summary(out.hk, perms=operm, alpha=0.2, pvalues=TRUE)|
\normalcolor

\end{enumerate}


\vspace{12pt}
\textbf{Interval estimates of QTL location} \vspace{6pt}
\nopagebreak

For the blood pressure phenotype, we've seen good evidence for QTL on
chromosomes 7 and 15.  Interval estimates of the location of QTL are
commonly obtained via 1.5-LOD support intervals, which may be
calculated via the function \verb-lodint-.  Alternatively, an
approximate Bayes credible interval may be obtained with
\verb-bayesint-.  

\begin{enumerate}
\addtocounter{enumi}{28}
\item To obtain the 1.5-LOD support interval and 95\% Bayes interval
  for the QTL on chromosome 7, type:

\usercolor
\verb|lodint(out.hk, chr=7)| \\
\verb|bayesint(out.hk, chr=7)| 
\normalcolor

The first and last rows define the ends of the intervals; the middle
row is the estimated QTL location.

\item It is sometimes useful to identify the closest flanking markers;
  use \verb-expandtomarkers=TRUE-:

\usercolor
\verb|lodint(out.hk, chr=7, expandtomarkers=TRUE)| \\
\verb|bayesint(out.hk, chr=7, expandtomarkers=TRUE)| 
\normalcolor

\item We can calculate the 2-LOD support interval and the 99\% Bayes
  interval as follows.

\usercolor
\verb|lodint(out.hk, chr=7, drop=2)| \\
\verb|bayesint(out.hk, chr=7, prob=0.99)| 
\normalcolor

\item The intervals for the chr 15 locus may be calculated as follows.

\usercolor
\verb|lodint(out.hk, chr=15)| \\
\verb|bayesint(out.hk, chr=15)| 
\normalcolor

\end{enumerate}

\vspace{12pt}
\textbf{QTL effects} \vspace{6pt}
\nopagebreak

We may obtain plots indicating the estimated effects of the QTL via
\verb-plotPXG-, which creates a dot plot, or \verb-effectplot-, which
plots the average phenotype for each genotype group.

\begin{enumerate}
\addtocounter{enumi}{32}
\item For \verb-plotPXG-, we must first identify the marker closest
  to the QTL peak.  Use \verb-find.marker-.

\usercolor
\verb|max(out.hk)| \\
\verb|mar <- find.marker(sug, chr=7, pos=47.7)| \\
\verb|plotPXG(sug, marker=mar)|
\normalcolor

Note that red dots correspond to inferred genotypes (based on a single
imputation). 


\item The function \verb-effectplot- uses the multiple imputation
  results from \verb-sim.geno-.

\usercolor
\verb|effectplot(sug, mname1=mar)|
\normalcolor

\item We may use \verb-effectplot- at a position on the ``grid''
  between markers, using \verb-"7@47.7"- to indicate the position at
  47.7~cM on chr~7.

\usercolor
\verb|effectplot(sug, mname1="7@47.7")|
\normalcolor

\item Similar plots may be obtained for the locus on chr 15.

\usercolor
\verb|max(out.hk, chr=15)| \\
\verb|mar2 <- find.marker(sug, chr=15, pos=12)| \\
\verb|plotPXG(sug, marker=mar2)| \\
\verb|effectplot(sug, mname1="15@12")|
\normalcolor

\item We may plot the joint effects of the two loci via
  \verb-plotPXG- as follows:

\usercolor
\verb|plotPXG(sug, marker=c(mar, mar2))| \\
\verb|plotPXG(sug, marker=c(mar2, mar))|
\normalcolor

\item The function \verb-effectplot- gives more readable figures in
  this case; it's often useful to look at it in both ways.

\usercolor
\verb|effectplot(sug, mname1="7@47.7", mname2="15@12")| \\
\verb|effectplot(sug, mname2="7@47.7", mname1="15@12")| 
\normalcolor

The two loci do not appear to interact.


\end{enumerate}

\vspace{12pt}
\textbf{Other phenotypes} \vspace{6pt}
\nopagebreak

By default in \verb-scanone-, we consider the first phenotype in the
input cross object.  Other phenotypes, include the parallel
consideration of multiple phenotypes, can be considered via the
argument \verb-pheno.col-.


\begin{enumerate}
\addtocounter{enumi}{38}
\item To analyze the second phenotype, refer to it by its numeric
  index, as follows.

\usercolor
\verb|out.hr <- scanone(sug, pheno.col=2, method="hk")|
\normalcolor

\item Alternatively, refer to a phenotype by its name:

\usercolor
\verb|out.bw <- scanone(sug, pheno.col="bw", method="hk")|
\normalcolor

\item You can also give a numeric vector of phenotype values.  This is
  useful for considering a transformed version of a phenotype, such as
  log body weight.

\usercolor
\verb|out.logbw <- scanone(sug, pheno.col=log(sug$pheno$bw), method="hk")|
\normalcolor

\item Use of vector of phenotype indices results in an object with
  multiple LOD score columns, one for each phenotype.

\usercolor
\verb|out.all <- scanone(sug, pheno.col=1:4, method="hk")|
\normalcolor

\item For this final case, it's important to note that the
  \verb-summary- function, by default, focuses solely on the first
  LOD score column.

\usercolor
\verb|summary(out.all, threshold=3)|
\normalcolor

Here, it looks at the first LOD score column and picks off the peaks
that are above 3, and then gives the LOD scores at that location for
the other three columns.  

To do the same thing but focusing on another column, use the argument
\verb-lodcolumn-. 

\usercolor
\verb|summary(out.all, threshold=3, lodcolumn=4)|
\normalcolor

\item Alternatively, use \verb-format="allpeaks"-, to get the maximum
  LOD score for each column, with a chromosome being shown if at least
  one of the LOD score column exceeds the threshold.

\usercolor
\verb|summary(out.all, threshold=3, format="allpeaks")|
\normalcolor

\item A third version of the output is obtained with
  \verb-format="allpheno"-, which gives one row per LOD peak and gives
  the LOD scores for all columns at each peak.

\usercolor
\verb|summary(out.all, threshold=3, format="allpheno")|
\normalcolor

\end{enumerate}


\vspace{12pt}
\textbf{Two-dimensional, two-QTL scans} \vspace{6pt}
\nopagebreak

Two-dimensional, two-QTL scans offer the opportunity to detect
interacting loci or to separate pairs of linked QTL.  Analysis is
performed with \verb-scantwo-, which is much like \verb-scanone-.


\begin{enumerate}
\addtocounter{enumi}{45}
\item For 2d scans, it's advantageous to run things at a coarser step
  size, by first re-running \verb-calc.genoprob-.

\usercolor
\verb|sug <- calc.genoprob(sug, step=2)|
\normalcolor

\item To perform a 2d scan for the blood pressure phenotype, use the
  following.  If you loaded the file \verb-"various.RData"- in
  step~\ref{various} on page~\pageref{various},
  you can skip this, as you already have the results.

\usercolor
\verb|out2 <- scantwo(sug, method="hk")|
\normalcolor

\item We may plot the results as follows.

\usercolor
\verb|plot(out2)|
\normalcolor

The upper-triangle contains interaction LOD scores, comparing the full
two-locus model to the additive two-locus model.  The lower-triangle
contains the ``full'' LOD scores, comparing the full two-locus model
to the null model.  Because of the clear evidence for QTL on
chromosomes 7 and 15, we see ``tails'' along those two chromosomes:
the two locus model with either chr 7 or chr 15 and anything else is clearly better
than the null model.  

\item It's best to replace the lower-triangle with the LOD score comparing the full model to
  the best single-QTL model, using either \verb:lower="cond-int": or
  \verb:lower="fv1": (the two are equivalent).

\usercolor
\verb|plot(out2, lower="fv1")|
\normalcolor

\item We can also look at the LOD scores comparing the additive
  two-QTL model to the best single-QTL model, using either
  \verb:upper="cond-add": or \verb:upper="av1":.

\usercolor
\verb|plot(out2, lower="fv1", upper="av1")|
\normalcolor

\item To assess significance, we need to do a permutation test.  This
  can be extremely time consuming.  The results were already loaded in
  step~\ref{various} on page~\pageref{various}, but here is the code (though here I cite \verb:n.perm=5:
  rather than \verb:n.perm=1000:, as I'd recommend).

\usercolor
\verb|operm2 <- scantwo(sug, method="hk", n.perm=5)|
\normalcolor

\item With the permutation results in hand, we can get a summary with
  p-values.

\usercolor
\verb|summary(out2, perms=operm2, alpha=0.2, pvalues=TRUE)|
\normalcolor

The pair of loci on 7 and 15 are clear.  They show no evidence for an
interaction.  There is some evidence for an additional locus on chr
12, with p=0.17.

\end{enumerate}


\vspace{12pt}
\textbf{Multiple-QTL analyses} \vspace{6pt}
\nopagebreak

After performing the single- and two-QTL genome scans, it's best to
bring the identified loci together into a joint model, which we then
refine from which we may explore the possibility of further QTL.  In
this effort, we work with ``QTL objects'' created by \verb-makeqtl-.
We fit multiple-QTL models with \verb-fitqtl-.  A number of additional
functions will be introduced below.

\begin{enumerate}
\addtocounter{enumi}{52}
\item Let's re-run \verb-calc.genoprob- so that we are working at a
  step size of 1~cM again.

\usercolor
\verb|sug <- calc.genoprob(sug, step=1)|
\normalcolor

\item First, we create a QTL object containing the loci on chr 7 and
  15.

\usercolor
\verb|qtl <- makeqtl(sug, chr=c(7,15), pos=c(47.7, 12), what="prob")|
\normalcolor

The last argument, \verb-what="prob"-, indicates to pull out the QTL
genotype probabilities for use in Haley-Knott regression.

\item We fit the two locus additive model as follows.

\usercolor
\verb|out.fq <- fitqtl(sug, qtl=qtl, method="hk")| \\
\verb|summary(out.fq)|
\normalcolor

A key part of the output is the ``drop one term at a time'' table,
which compares the fit of the two-QTL model to the reduced model in
which a single QTL is omitted.  

\item We may obtain the estimated effects of the QTL via
  \verb-get.ests=TRUE-.  We use \verb-dropone=FALSE- to suppress the
  drop-one-term analysis.

\usercolor
\verb|summary(fitqtl(sug, qtl=qtl, method="hk", get.ests=TRUE, dropone=FALSE))|
\normalcolor

Since this is an intercross, we obtain estimates of the additive
effect and dominance deviation for each locus.

\item To assess the possibility of an interaction between the two
  QTL, we may fit the model with the interaction, indicated via a
  model ``formula''.  The QTL are referred to as \verb-Q1- and
  \verb-Q2- in the formula, and we may indicate the interaction in a
  couple of different ways.

\usercolor
\verb|out.fqi <- fitqtl(sug, qtl=qtl, method="hk", formula=y~Q1*Q2)| \\
\verb|out.fqi <- fitqtl(sug, qtl=qtl, method="hk", formula=y~Q1+Q2+Q1:Q2)| \\
\verb|summary(out.fqi)|
\normalcolor

We don't have time to cover the use of such formulas in any detail
here.  Note that there is no evidence for an interaction.

\item Another way to assess interactions is with the function
  \verb-addint-, which adds one interaction at a time, in the context
  of a multiple-QTL model.  This is most useful when there are more
  than two QTL being considered.

\usercolor
\verb|addint(sug, qtl=qtl, method="hk")|
\normalcolor

\item The locations of the two QTL are as estimated via the single-QTL
  scan.  We may refine our estimates of QTL location in the context of
  the multiple-QTL model via \verb-refineqtl-.  This function
  uses a greedy algorithm to iteratively refines the locations of the
  QTL, one at a time, at each step seeking to improve the overall fit.

\usercolor
\verb|rqtl <- refineqtl(sug, qtl=qtl, method="hk")| \\
\verb|rqtl|
\normalcolor

The location of each QTL changed slightly, and the overall LOD score
increased by 0.03.

\item We can re-run \verb-fitqtl- to get the revised drop-one-term
  table.

\usercolor
\verb|summary(out.fqr <- fitqtl(sug, qtl=rqtl, method="hk"))|
\normalcolor

\item The \verb-plotLodProfile- function plots LOD profiles obtained
  during the call to \verb-refineqtl-.  These give one-dimensional
  views of the precision of QTL localization, in the context of the
  multiple-QTL model.

\usercolor
\verb|plotLodProfile(rqtl)|
\normalcolor

For each position on the curve for the chr 7 QTL, we compare the
two-QTL model with the chr 7 locus in varying position but with the
chr 15 locus fixed at its estimated position, to the single-QTL model
with just the chr 15 locus.  The chr 15 curve is similar.

These are actually slightly lower than the curves obtained from
the single-QTL analysis with \verb-scanone-.

\usercolor
\verb|plot(out.hk, chr=c(7,15), col="red", add=TRUE)|
\normalcolor

\item The function \verb-addqtl- is used to scan for an additional QTL
  to be added to the model.

\usercolor
\verb|out.aq <- addqtl(sug, qtl=rqtl, method="hk")|
\normalcolor

The biggest peaks are on chr 8 and 12, but nothing is particularly
exciting.

\usercolor
\verb|plot(out.aq)|
\normalcolor

There is also a function \verb-addpair-, for scanning for a pair of
QTL to be added.


\item Finally, we consider the function \verb-stepwiseqtl-, which is
  our fully automated stepwise algorithm to optimize the penalized LOD
  scores of Manichaikul et al. (2009).  We first need to calculate the
  appropriate penalties from the two-dimensional permutation results.

\usercolor
\verb|print(pen <- calc.penalties(operm2))|
\normalcolor

We then run \verb-stepwiseqtl-, using \verb-max.qtl=5-.  It will
perform forward selection to a model with 5 QTL, followed by backward
elimination, and will report the model giving the largest penalized
LOD score.  The output is a QTL object.

\usercolor
\verb|out.sq <- stepwiseqtl(sug, max.qtl=5, penalties=pen, method="hk", verbose=2)| \\
\verb|out.sq|
\normalcolor

With \verb-verbose=2-, we get an indication of the location of the QTL
at each step.  

The result is exactly the model that we had after \verb-refineqtl-.

\end{enumerate}



\vspace{12pt}
\textbf{References} \vspace{6pt}
\nopagebreak

\begin{hanging}

\item Sugiyama F, Churchill GA, Li R, Libby LJM, Carver T, Yagami K-I,
  John SWM, Paigen B (2002) QTL associated with blood pressure, heart
  rate, and heart weight in CBA/CaJ and BALB/cJ mice.  Physiol
  Genomics 10:5--12

\item Broman KW, Sen \'S (2009) A guide to QTL mapping with
    R/qtl.  Springer

\item 	Manichaikul A, Moon JY, Sen \'S, Yandell BS, Broman KW (2009)
  A model selection approach for the identification of quantitative
  trait loci in experimental crosses, allowing epistasis. Genetics
  181:1077--1086 

\item Dalgaard P (2008) Introductory statistics with R, 2nd edition.
  Springer

\item Venables WN, Ripley BD (2002) Modern applied statistics with S,
  4th edition. Springer


\end{hanging}


\end{document}
