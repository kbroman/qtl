\documentclass[11pt]{article}
\setlength{\topmargin}{-.5in}
\setlength{\textheight}{23.5cm}
\setlength{\textwidth}{17.0cm}
\setlength{\oddsidemargin}{.025in}
\setlength{\evensidemargin}{.025in}
\setlength{\textwidth}{6.25in}
\usepackage{amsmath}
\usepackage{graphicx}
\usepackage{verbatim}   % useful for program listings
\usepackage{color}      % use if color is used in text
\usepackage{subfigure}  % use for side-by-side figures
\usepackage{Sweave}
\title { Chapter 12 - automated mqm modeling via analysis of deviance }
\author { Danny Arends, Pjotr Prins }
\begin {document}
\maketitle
\clearpage
\section{MQM introduction}
In this chapter we will go into the automatic mapping of QTLs using the mqm method developed by Janssen et al. 1994\cite{jansen94}. This method has been added to the R/qtl\cite{broman09}\cite{broman03} package in 2009. The method consists of three main parts:\\
1) Missing data augmentation\\
2) Backward model selection using genetic markers as cofactors\\
3) QTL (interval) mapping using the 'most informative' model\\
The method internally controls false discovery rates and lets users test extensive QTLmodels bij elimation of non-significant cofactors.
In this chapter we will focus on the basic functions of mqm, we assume the reader knows howto load his data into R using the R/qtl $read.cross()$ 
function.
\section{Missing data augmentation}
First lets go into the augmentation of missing data. we start by loading the qtl library and simulating some F2 intercross data. We use plot.geno to visualze the genotypes of the individuals. we see \~2\% missing values in white. The other colors stand for genotypes at a certain position for a certain animal(individual).
\begin{Schunk}
\begin{Sinput}
> library(qtl)
> data(map10)
> mycross <- sim.cross(map10, type = "f2", n.ind = 100, missing.prob = 0.02)
\end{Sinput}
\end{Schunk}
\begin{figure}[h]
\begin{Schunk}
\begin{Sinput}
> plot.geno(mycross)
\end{Sinput}
\end{Schunk}
\includegraphics{chapter12-002}
\caption{Plot.Geno showing the genotypes of 100 individuals with around 2\% missing}
\end{figure}
Before we can use the mqm function, the data should be complete (no missing markers). We thus have to fill in the blanks. We have two options to do this, augmentation or fill.geno. The $fill.geno()$ function fills in the most likely markers using single or multiple imputations
to estimate the missing markersgenotypes. The first option augmentation was designed for mqm and the resulting cross object is not suited for scan.one or cim anymore. This missing data augmentation routine $mqmaugment()$ can fill in missing genotypes for us. For each missing marker it fills in all possible markers and calculates their likelyhood. When they are more likely than the $augment\_aboveprob$ parameter the augmented individual is stored in the new crossobject. The mqmaugment() function can only handle 1 phenotype, but this will be inproved upon soon. The important parameters are:
$cross, pheno.col, maxaugind, augment\_aboveprob$ and verbose set this to TRUE to see what happens. $Maxind$ sets a maximum to the size of the dataset the default is usually good enough (60 is around 20 missing F2 markers per individual or 30 BC markers). The augmentation routine does an all or nothing filling of the missing markers. The individual is expanded untill the $augment\_aboveprob$: Augment genotypes that are above a probability of occurring (ignore lower probabilities). Setting this value too high may result in dropping individuals entirely.
Lets start by simulating a dataset with some missing markers (2\%) and discuss howto use the augmentation routine:
\begin{Schunk}
\begin{Sinput}
> augmentedcross <- mqmaugment(mycross, augment_aboveprob = 1)
\end{Sinput}
\begin{Soutput}
Starting C-part of the data augmentation routine
F2 cross
Convert codes R/qtl -> MQM
Filling the chromosome matrix
Calculating relative genomepositions of the markers
Estimating recombinant frequencies
\end{Soutput}
\end{Schunk}
\begin{figure}[h]
\begin{Schunk}
\begin{Sinput}
> plot.geno(augmentedcross)
\end{Sinput}
\end{Schunk}
\includegraphics{chapter12-004}
\caption{Plot.Geno showing the genotypes of 100 individuals with no missing values (using mqmaugment as fill.geno)}
\end{figure}
When we use a larger $augment\_aboveprob$ (so more unlikely genotypes are also consideren we see the dataset expanding in size.
This expanded dataset can only be analysed correctly with the mqm routine, because scanone will treat duplicated individuals
as new individuals (that improve power)\cite{Dempster77}. Mqm does not do this it treats duplicated individuals together as a single entity, but because the algorithm is also aware of the other possible genotypes resulting QTLprofiles will be more accurate\cite{jansen93}.
\\
\begin{figure}[h]
\begin{Schunk}
\begin{Sinput}
> augmentedcross <- mqmaugment(mycross, augment_aboveprob = 10)
\end{Sinput}
\begin{Soutput}
Starting C-part of the data augmentation routine
F2 cross
Convert codes R/qtl -> MQM
Filling the chromosome matrix
Calculating relative genomepositions of the markers
Estimating recombinant frequencies
\end{Soutput}
\begin{Sinput}
> plot.geno(augmentedcross)
\end{Sinput}
\end{Schunk}
\includegraphics{chapter12-005}
\caption{Plot.Geno showing the genotypes of 100 individuals with no missing values (using mqmaugment and letting it expand the dataset)}
\end{figure}
\clearpage
\section{Mqm modeling and mapping}
We start bij analyzing the hyperset data this can be loaded into memory by using $data(hyper)$. This set is an F2 mouse set with a two phenotypes bp, bloodpressure and Sex. \texttt{qtl.mqm} help page into a \LaTeX{} document
\\
\begin{Schunk}
\begin{Sinput}
> data(hyper)
> colors <- c("Black", "Green")
> lines <- c(2, 1)
> h_no_missing <- mqmaugment(hyper, augment_aboveprob = 1)
\end{Sinput}
\begin{Soutput}
Starting C-part of the data augmentation routine
Back cross (BC)
Convert codes R/qtl -> MQM
Filling the chromosome matrix
Calculating relative genomepositions of the markers
Estimating recombinant frequencies
\end{Soutput}
\begin{Sinput}
> result <- mqm(h_no_missing)
\end{Sinput}
\begin{Soutput}
Back cross (BC)
Convert codes R/qtl -> MQM
Receiving the chromosome matrix from R
Calculating relative genomepositions of the markers
Estimating recombinant frequencies
Initialize Frun and informationcontent to 0.0
Calculating relative genomepositions of the markers
Estimating recombinant frequencies
After dropping of uninformative cofactors
Calculating relative genomepositions of the markers
Estimating recombinant frequencies
Analysis of data finished
\end{Soutput}
\begin{Sinput}
> result_compare <- scanone(h_no_missing)
\end{Sinput}
\end{Schunk}
If we then plot the results from both QTLscans (Black=mqm,Green=scanone):
\begin{figure}[h]
\begin{Schunk}
\begin{Sinput}
> plot(result, result_compare, col = colors, lwd = lines)
\end{Sinput}
\end{Schunk}
\includegraphics{chapter12-007}
\caption{QTlprofiles of the trait bp (blood pressure) in an experiment with 250 mice using mqm and scanone (black,green). We see one to 
one correspondance between the scanone function and the mqm routine when using no parameters}
\end{figure}
We can now use two approaches:\\
1) Start building a model by hand or,\\
2) Use unsupervised backward selection on a large number of markers\\
We will start by first building the model by hand. We see the big peek on chromosome 4 at 30 centimorgan, so lets account for that by setting a cofactor at the marker nearest to the peek on chromosome 4. This is done by the following steps:
\\
\begin{Schunk}
\begin{Sinput}
> summary(result)
\end{Sinput}
\begin{Soutput}
          chr pos (Cm) QTL bp  Info QTL*INFO
c1.loc80    1       80  3.534 0.884    3.123
c2.loc60    2       60  1.448 0.902    1.306
c3.loc25    3       25  0.612 0.909    0.557
c4.loc30    4       30  7.059 0.923    6.518
c5.loc70    5       70  1.428 0.932    1.331
c6.loc25    6       25  1.755 0.949    1.665
c7.loc25    7       25  0.361 0.939    0.339
c8.loc65    8       65  1.674 0.904    1.513
c9.loc70    9       70  1.464 0.928    1.359
c10.loc10  10       10  0.134 0.907    0.121
c11.loc40  11       40  0.703 0.904    0.636
c12.loc0   12        0  0.442 0.928    0.410
c13.loc0   13        0  0.344 0.845    0.291
c14.loc10  14       10  0.324 0.937    0.304
c15.loc20  15       20  2.204 0.924    2.037
c16.loc50  16       50  0.945 0.937    0.885
c17.loc55  17       55  0.136 0.862    0.117
c18.loc10  18       10  1.369 0.888    1.216
c19.loc0   19        0  0.700 0.950    0.665
c20.loc40  20       40  2.044 0.910    1.860
\end{Soutput}
\begin{Sinput}
> find.marker(h_no_missing, chr = 4, pos = 30)
\end{Sinput}
\begin{Soutput}
[1] "D4Mit164"
\end{Soutput}
\begin{Sinput}
> toset <- which.marker(h_no_missing, "D4Mit164")
\end{Sinput}
\begin{Soutput}
Marker D4Mit164 is number 46 
\end{Soutput}
\begin{Sinput}
> cofactorlist <- mqmcofactors(h_no_missing, toset)
\end{Sinput}
\end{Schunk}
\begin{figure}[h]
\begin{Schunk}
\begin{Sinput}
> result <- mqm(h_no_missing, cofactorlist, plot = T)
\end{Sinput}
\begin{Soutput}
Back cross (BC)
Convert codes R/qtl -> MQM
Receiving the chromosome matrix from R
Calculating relative genomepositions of the markers
Estimating recombinant frequencies
Initialize Frun and informationcontent to 0.0
Calculating relative genomepositions of the markers
Estimating recombinant frequencies
After dropping of uninformative cofactors
Calculating relative genomepositions of the markers
Estimating recombinant frequencies
Analysis of data finished
\end{Soutput}
\end{Schunk}
\includegraphics{chapter12-009}
\\
\caption{QTlprofiles of the trait bp (blood pressure) in an experiment with 250 mice using mqm. When we use a cofactor at chromosome 4 (D4Mit164), which is kept in the model. We see the LOD score (evidence) for a second QTL on chromosome 1 increasing. Also note the QTL BP and the QTL*INFO. the info parameter is an estimation of the informationcontent present in the marker underlying the qtl.}
\end{figure}
The previous example shows howto set a single marker as a cofactor, and do a mqm qtl scan, the marker isn't dropped and we know that it passed initial rough thresholding to account for the user defined alpha level. We accounted here for the large peek on chromosome 4 of the mice with high blood pressure. Comparing this back to the original scanone using:
\begin{figure}[h]
\begin{Schunk}
\begin{Sinput}
> plot(result, result_compare, col = colors, lwd = lines)
\end{Sinput}
\end{Schunk}
\includegraphics{chapter12-010}
\caption{QTlprofiles of the trait bp (blood pressure) in an experiment with 250 mice using mqm and scanone (black, green). When using a cofactor at chromosome 4 (D4Mit164) to account for variation explained by that marker.}
\end{figure}
The second peek on chromosome 1 at \~70 centimorgan becomes higher, so lets try adding that one to the model also and see if the model with both cofactors is even better at explaining the phenotype. We combine which.marker with find.marker for easy coding, we combine ($c$) the new cofactornumber with the one we already had in the toset variable.
\\
\begin{Schunk}
\begin{Sinput}
> summary(result)
\end{Sinput}
\begin{Soutput}
          chr pos (Cm) QTL bp  Info QTL*INFO
c1.loc70    1       70 5.2241 0.946   4.9398
c2.loc60    2       60 1.3684 0.901   1.2333
c3.loc25    3       25 0.8108 0.909   0.7369
c4.loc30    4       30 7.9020 0.924   7.3047
c5.loc70    5       70 1.4894 0.932   1.3887
c6.loc25    6       25 1.5220 0.949   1.4441
c7.loc25    7       25 0.2024 0.939   0.1902
c8.loc60    8       60 1.3239 0.940   1.2447
c9.loc65    9       65 1.2612 0.896   1.1302
c10.loc15  10       15 0.0940 0.947   0.0890
c11.loc40  11       40 0.8636 0.904   0.7804
c12.loc0   12        0 0.3386 0.928   0.3143
c13.loc0   13        0 0.1989 0.845   0.1680
c14.loc10  14       10 0.3804 0.937   0.3566
c15.loc20  15       20 2.4776 0.924   2.2887
c16.loc50  16       50 0.6714 0.938   0.6295
c17.loc5   17        5 0.0825 0.944   0.0779
c18.loc10  18       10 0.8184 0.885   0.7241
c19.loc45  19       45 0.4146 0.882   0.3656
c20.loc40  20       40 1.7845 0.911   1.6255
\end{Soutput}
\begin{Sinput}
> toset <- c(toset, which.marker(h_no_missing, find.marker(h_no_missing, 
+     1, 70)))
\end{Sinput}
\begin{Soutput}
Marker D1Mit218 is number 12 
\end{Soutput}
\begin{Sinput}
> cofactorlist <- mqmcofactors(h_no_missing, toset)
\end{Sinput}
\end{Schunk}
\begin{figure}[h]
\begin{Schunk}
\begin{Sinput}
> result <- mqm(h_no_missing, cofactorlist, plot = T)
\end{Sinput}
\begin{Soutput}
Back cross (BC)
Convert codes R/qtl -> MQM
Receiving the chromosome matrix from R
Calculating relative genomepositions of the markers
Estimating recombinant frequencies
Initialize Frun and informationcontent to 0.0
Calculating relative genomepositions of the markers
Estimating recombinant frequencies
After dropping of uninformative cofactors
Calculating relative genomepositions of the markers
Estimating recombinant frequencies
Analysis of data finished
\end{Soutput}
\end{Schunk}
\includegraphics{chapter12-012}
\caption{QTlprofiles of the trait bp (blood pressure) in an experiment with 250 mice using mqm. When we use a cofactor at chromosome 4 (D4Mit164) and at Chromosome 1 () we see both are included into the model.}
\end{figure}
\begin{figure}[h]
\begin{Schunk}
\begin{Sinput}
> plot(result, result_compare, col = colors, lwd = lines)
\end{Sinput}
\end{Schunk}
\includegraphics{chapter12-013}
\caption{QTlprofiles of the trait bp (blood pressure) in an experiment with 250 mice using mqm and scanone (black, green). When we use a cofactor at chromosome 4 (D4Mit164) and at Chromosome 1 () to account for variation explained by those two markers.}
\end{figure}
\\
Because mqm internally checks that no markers are included that are less significant than the alpha level specified. This marker is also informative enough to be included into the model, we can continue this process of adding
cofactors untill there are no more informative markers that can be included. This could be very time consuming in the case of many QTLs underlying your trait, we can also explore our data using the second approach "unsupervised backward selection" on a large number of markers, and set cofactors every other marker. The algorithm will analyse all the markers and if found to be not informative enough drop them from the model. After selection it will scan the chromosome using the model created from the cofactors. We should setthe plot parameter to TRUE ($plot=T$) this way we also get a graphical overview of the model that was used for scanning. So lets set a cofactor at every fifth marker and see which chromosomes can be implicated in the high bloodpressure from which our mice suffer. We can set cofactors by using the $mqmcofactorsEach()$ function\\
\begin{figure}[h]
\begin{Schunk}
\begin{Sinput}
> cofactorlist <- mqmcofactorsEach(h_no_missing, 5)
> result <- mqm(h_no_missing, cofactorlist, plot = T)
\end{Sinput}
\begin{Soutput}
Back cross (BC)
Convert codes R/qtl -> MQM
Receiving the chromosome matrix from R
Calculating relative genomepositions of the markers
Estimating recombinant frequencies
Initialize Frun and informationcontent to 0.0
Calculating relative genomepositions of the markers
Estimating recombinant frequencies
After dropping of uninformative cofactors
Calculating relative genomepositions of the markers
Estimating recombinant frequencies
Analysis of data finished
\end{Soutput}
\end{Schunk}
\includegraphics{chapter12-014}
\caption{QTlprofiles of the trait bp (blood pressure) in an experiment with 250 mice using mqm. We have set cofactors at every fifth marker, and backward selected them to obtain the most likely model (based on out selection of cofactors).
}
\end{figure}
Comparing back the result to the original scanone result we see quite some striking differences:
\\
\begin{figure}[h]
\begin{Schunk}
\begin{Sinput}
> plot(result, result_compare, col = colors, lwd = lines)
\end{Sinput}
\end{Schunk}
\includegraphics{chapter12-015}
\caption{QTlprofiles of the trait bp (blood pressure) in an experiment with 250 mice using mqm and scanone (black, green). We have set cofactors at every fifth marker for mqm, and backward selected them to obtain the most likely model (based on selection of cofactors, alpha = 0.02). We see some striking differences between the resulting QTLprofiles.}
\end{figure}
\\
This leads to a lot of hits and multiple hits on each chromosome, because some cofactors are placed too close to eachother
However we see in fig 10 that at an alpha of 0.02 according to mqm chromosomes 1,2,4,5,6 and (15?) are implicated using this method
This is a fairly extensive model, and lowering the significance level from 0.02 to 0.002 could yield a more comprehensable model.
\begin{figure}[h]
\begin{Schunk}
\begin{Sinput}
> result <- mqm(h_no_missing, cofactorlist, alfa = 0.002, plot = T)